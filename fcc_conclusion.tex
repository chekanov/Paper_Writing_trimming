\section{Conclusions}
The studies presented in this paper show that the reconstruction of jet substructure 
variables for future particle colliders will benefit from small cell sizes of the hadronic calorimeters. 
This conclusion was obtained using the realistic \GEANTfour\ simulation of calorimeter response combined with reconstruction of 
calorimeter clusters used as inputs for jet reconstruction. 
Hadronic calorimeters that use the cell sizes of 20~$\times $~20~cm$^2$ ($\Delta \eta \times \Delta \phi = 0.087\times 0.087$) 
are least performant for almost every 
substructure variable considered in this analysis, for jet transverse momenta between 2.5 and 10~TeV. 
Such cell sizes are similar to 
those used for the ATLAS and CMS detectors at the LHC. 
In terms of reconstruction of physics-motivated quantities  
used for jet substructure studies, the  performance 
of a  hadronic calorimeter  with 
$\Delta \eta \times \Delta \phi = 0.022\times0.022$ ($5 \times 5$~$\mathrm{cm}^2$ cell size) is, in most cases,
better than for a detector with  $0.087\times 0.087$ cells.

Thus this study confirms the  HCAL geometry of the SiFCC detector~\cite{Chekanov:2016ppq},
with the $\Delta \eta \times \Delta \phi = 0.022\times0.022$ HCAL cells.
It also confirms the HCAL design of the baseline FCC-hh~\cite{fcc1,fcc2} detector with
$\Delta \eta \times \Delta \phi = 0.025\times0.025$ HCAL cells.

It interesting to note that,  for very boosted jets with transverse momenta close to 20~TeV, further decrease of cell size to $\Delta \eta \times \Delta \phi = 0.0043\times0.0043$ did not 
 definitively show a further improvement in performance. 
 This result needs to be understood in terms of various types of simulations and 
different options for reconstruction of the calorimeter clusters.
